\chapter{基于自底向上框架的视频片段检索方法}

\section{引言}

\section{视频片段检索}

\section{实验设置与性能对比}

\subsection{视频片段检索数据集}

\noindent\textbf{\kaishu{基于语句的视频片段检索}}:我们在以下三个数据集上进行评估:

\noindent\textbf{TACoS}~\cite{regneri2013grounding}:它一共包含127个视频和17344个文本与视频序列对(样本)。我们参考现有的标准数据集划分~\cite{gao2017tall},将其中50\%的样本作为训练集,25\%的样本作为验证集,25\%的样本作为测试集。每个样本中视频的平均长度为5分钟。


\noindent\textbf{Charades-STA}~\cite{gao2017tall}:它一共包含12408个文本与视频序列对作为训练集,3720个文本与视频序列对作为测试集。每个样本中视频的平均长度为30秒。


\noindent\textbf{ActivityNet Captions}~\cite{krishna2017dense}:它是目前为止最大、最丰富的数据集,一共包含19209个视频。我们参考现有的工作~\cite{yuan2019find}, 使用37421个文本与视频序列作为训练集,17505个文本与视频序列作为测试集。每个样本中视频的平均长度为2分钟。


\noindent\textbf{\kaishu{基于视频的视频片段检索}}:我们在以下数据集上进行评估:

\noindent\textbf{ActivityNet-VRL}~\cite{feng2018video}:它是目前唯一公开发布的数据集。它对动作识别数据集ActivityNet~\cite{caba2015activitynet}共200个类别的视频进行了重组,其中任意选取160个类别对应的视频作为训练集,20个类别对应的视频作为验证集,以及剩余20个类别对应的视频作为测试集。这种零样本式的数据集划分能够评估模型的泛化能力。在训练阶段,查询视频和引用视频是随机选取的。在测试阶段,查询视频和引用视频是固定的。

\subsection{评价指标}


\noindent\textbf{\kaishu{基于语句的视频片段检索}}:我们参考现有工作,使用下列两种通用的评价指标:

\noindent\textbf{R@N, IoU@$\theta$}:在测试集中,每个样本预测分数最高的n个的结果重叠度(Intersection-over-Union,IoU)大于$\theta$的百分比。由于自底向上框架的特性,我们仅考虑$N=1$。

\noindent\textbf{mIoU}:测试集中所有测试样本的平均的重叠度。


\noindent\textbf{\kaishu{基于视频的视频片段检索}}:我们使用以下评价指标:

\noindent\textbf{mAP@1}:在不同阈值下最高预测结果的平均精度均值(mAP)。

\section{本章小结}

