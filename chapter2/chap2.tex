\chapter{相关研究综述}

本章将就感兴趣零样本物体识别、图像场景图生成、图像描述语句生成、视频片段检索和图像视觉问答几方面的相关工作和本文的关系进行综述。

本文提出的算法和其相关工作的具体细节和对比将在之后各章节中展示。


\section{零样本物体识别}

\subsection{零样本学习}
零样本物体识别的主流方法是基于类别属性的物体识别~\cite{farhadi2009describing,lampert2009learning,romera2015embarrassingly,norouzi2014zero,demirel2017attributes2classname,jiang2017learning}:这类方法通常将类别属性看成是一个共同语义空间的中间特征,从而实现对不同类别之间的语义迁移。为了实现零样本,流行的类别属性的方法~\cite{frome2013devise,}都是通过


\subsection{域偏移问题}

\subsection{对抗生成网络}




\section{图像场景图生成}

\subsection{场景图生成}
 

\subsection{多智能体梯度策略}






\section{图像描述语句生成}


\kaishu{空间注意力机制}


\kaishu{语义注意力机制}


\kaishu{多层注意力机制}







\section{视频片段检索}

\subsection{基于文本的视频片段检索}


\subsection{基于视频的视频片段检索}


\subsection{自上向下框架与自底向上框架}







\section{图像视觉问答}


\subsection{视觉问答模型的文本偏差}


\subsection{视觉问答模型的特性}

\kaishu{视觉可解释性}


\kaishu{文本敏感性}