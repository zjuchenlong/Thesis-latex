\chapter{基于属性保持的零样本物体分类方法}


\section{问题描述}
零样本物体识别(Zero-Shot Recogintion, ZSR)或零样本学习(Zero-Shot Learning, ZSL)是为了能够对训练过程未见过的新类别图像进行分类识别。目前,关于零样本学习问题的难点,学界的共识是如何将已见类别的知识迁移到未见类别上。尽管到目前为止,已经有非常多的零样本分类方法,这些方法都是依据一些非常简单和直观的机制。例如,虽然“浣熊”这个类别在训练的时候没有见过,但是我们仍然可以识别出浣熊的图像,通过检查浣熊这个类别特有的一些属性特征,像“有条纹的尾巴”~\cite{farhadi2009describing,lampert2009learning,zhang2013attribute,li2010object}、“像狐狸的外观”~\cite{torresani2010efficient,li2010object}、以及“浣熊”这个类别的语义信息~\cite{pennington2014glove,mikolov2013distributed}。这些属性特征通常在训练阶段被建模,然后期待在测试阶段中可以在所有的类别(已见类别和未见类别)中共享。经过数十年的发展,目前的零样本学习框架已经从初始基于属性分类器的模型~\cite{lampert2009learning}发展到基于嵌入映射的模型~\cite{akata2015label,frome2013devise,weston2010large}。






\section{属性保持的对抗网络学习}


\section{实验设置与性能分析}
\subsection{零样本物体分类数据集}
\textbf{CUB}~\cite{wah2011caltech}:全称是Caltech-UCSD-Birds 200-2011数据集。它是一个细粒度鸟类别分类数据集,总共包含11788张来自200个细粒度类别的鸟图像,并且每张图像有312个语义属性标注。其中训练集包含150个已见类别的7057张图像,测试集包含150个已见类别的1764张图像和50个未见类别的2967张图像。

\textbf{SUN}~\cite{patterson2012sun}:全称是SUN attribute数据集。它是一个细粒度场景分类数据集,总共包含14340张来自717个场景类别的场景图像,并且每张图像有102个语义属性标注。其中训练集包含645个已见类别的10320张图像,测试集包含645个已见类别的2580张图像和72个未见类别的1440张图像。

\textbf{AWA}~\cite{lampert2009learning}:全称是Animals with Attributes数据集。它是一个动物类别分类数据集,总共包含30475张来自50个类别的动物图像,并且每张图像有85个语义属性标注。其中训练集包含40个已见类别的23527张图像,测试集包含40个已见类别的5882张图像和10个未见类别的7913张图像。由于原始AWA数据集图像版权的问题,我们这里的AWA数据集实际上使用的是AWA2~\cite{xian2017zero}.

\textbf{aPY}~\cite{farhadi2009describing}:全称是Attribute Pascal and Yahoo数据集。它是一个通用的物体分类数据集,总共包含12051张来自32个类别,并且每张图像有64个语义属性标注。其中训练集包含20个已见类别5932张图像,测试集包含20个已见类别1483张图像和12个未见类别的7924张图像。

为了公平地和其他模型进行比较,我们使用Xian等人~\cite{xian2017zero}提供的类别嵌入映射向量,其中每个嵌入映射向量都经过$l_2$范数进行归一化。


\subsection{实验设定与零样本物体分类评价指标}
\noindent{\kaishu{实验设定}}:
为了评估模型对零样本物体分类的结果,我们采用三种实验设定:
\begin{asparaenum}
\item U$\to$U: 测试图像的类别和可以预测的类别都只是未见类别;

\item S$\to$T: 测试图像的类别是未见类别,但是可以预测的类别是未见类别和已见类别的总和;

\item U$\to$T: 测试图像的类别和可以预测的类别都是未见类别和已见类别的总和。
\end{asparaenum}
通常,U$\to$U被称为传统型零样本分类,而U$\to$T被称为通用型零样本分类。

\noindent{\kaishu{评价指标}}:
我们参考现有的文献~\cite{xian2017zero},常用的每类平均准确率作为评价指标。对于通用型零样本分类,我们另外使用常用的$H$作为主要的评价指标,其中$H$是已见类别$L_s$的准确率($Acc_{S\rightarrow T}$)和未见类别$L_u$的准确率($Acc_{U\rightarrow T}$)的调和平均数:
\begin{equation} \label{ch3:equ:H}
H = 2\times Acc_{S\rightarrow T}\times Acc_{U\rightarrow T} /(Acc_{S\rightarrow T}+Acc_{U\rightarrow T})
\end{equation}

\subsection{网络模型与参数设置}
\noindent{\kaishu{网络结构}}:整个网络结构都是端到端地直接进行训练。其中映射网络$E$是基于ResNet-101~\cite{he2016deep},输入图像的大小是$224\times224\times3$。映射网络$F$是基于AlexNet~\cite{krizhevsky2012imagenet},然后附加上两层额外的全连接层。重建网络$G$采用类似于生成器~\cite{dosovitskiy2016generating}的结构,通过五个连续的反卷积和非线性操作(leaky ReLU)将向量特征转换成三维卷积特征。


\noindent{\kaishu{参数设置}}:对于本章所有的实验,训练图像都将短边放缩到256个像素。参照AlexNet~\cite{krizhevsky2012imagenet},我们采用了增大十倍训练图像的数据增强方式。为了提升训练速度,映射网络$E$中ResNet-101部分参数始终保持固定,映射网络$F$的参数初始化采用预训练好的AlexNet的参数,重建网络$G$的参数初始化用预训练好的生成器~\cite{dosovitskiy2016generating}。剩余的所有参数都是用MSRA的随机初始化~\cite{he2015delving}。初始的学习率设置为$1e^{-4}$,然后当loss不下降时,学习率降低10倍。


\section{本章小结}